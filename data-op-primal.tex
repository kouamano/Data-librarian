\subsection{文字列処理}
文字列処理の中心はパターンマッチおよび文字(列)書き換えの技術である。
パターンとは、特定の文字列または文字列集合を表すもので、多くの場合、このパターンそのものが文字列として表現される。
とくに、文字列集合を一つの文字列で表す方式を正規表現または正則表現と言う。
現在では、正規表現はライブラリ(エンジン)として纏められ、共通ライブラリが複数のユーティリティーで利用されている。
Perl互換正規表現ライブラリ(PCRE)と鬼車(Oniguruma)は代表的な正規表現ライブラリである。

一方で、バッカス・ナウア記法(BNF)は、文法定義のためのメタ言語である。すなわち与えられた文字列がその言語において受け入れられるかを決定するための表現である。

さらに、マークアップ言語もマークアップ言語に対する定義を表すスキーマ言語も、メタ言語として捉えることが可能である。

上記のいずれの表現も、文字列パターン定義および文字列処理定義に利用可能である。
\begin{breakbox}$\Rightarrow$
正規表現、BNF(文法定義記法)、マークアップ言語、に対するユーティリティーに何があるか調べてみよう。
\end{breakbox}

\subsection{数値処理}
数値処理は、有限精度の値の計算結果を、有限精度の値として出力する。
数値の精度はハードウエアによって決まるが、有限精度の値を配列させることにより任意精度の計算も可能となる。
この際の精度とは、純粋に表現可能なビット数である。
さらに、表現されているビット列が、どの程度確からしいかを表す言葉として、確度というものがある。
すなわち、ある数値が表現されいている時、この裏には精度や確度といった情報が隠れていることになる。

基本的に、実際の数値計算においては、計算を重ねるごとに確からしさが減少する。

\subsection{数式処理}
数式処理とは、数値処理を含むより広義な、Symbolic Computation System や、Computer Algebra System 等の概念を含むもので、実装としては、Mathematica、Mapleといった商用アプリケーション、Maximaのようなオープンソースが存在する。

Symbolic Computation System は、記号(symbol)を、何らかの変数記号として、数値の代入を行うことなく可能な処理を行う。例えば、代数処理として$a - a$の表現を$0$として評価する。この例では記号に対して代数処理が行われており、前述のアプリケーションに代表されるシステムにおいても、記号的な代数システムが用いられている。

\begin{breakbox}$\Rightarrow$
テスト用。
\end{breakbox}
