データ処理とは、計算機が扱うことのできる情報の変換である。
より詳細には、処理対象と処理命令の定義が記述された表現を、その表現以外の知識(ルール)を用い、表現が変化しなくなるまで評価を続けることであり、すなわち、端的には「表現評価系を用いた表現の変換」と言える。
この解釈では「処理命令」(=プログラム)と「処理対象」(=データ)は分離が曖昧であり、計算機科学としてはどちらも「処理対象」とすることがふさわしい。
このことは、より抽象的かつ厳密に万能チューリングマシンとして定義可能である。
\subsection{チューリングマシン}
チューリングマシンとは、アラン・チューリングによる仮想機械の仕組みであり、現実の計算機はこれを物理的に実装したものであり、次の要素から成る。
\begin{itemize}
\item 状態(有限)を記憶するメモリ
\item 記号(有限)列(無限)を記録するテープ
\item メモリとテープを読み書きするデバイス
\end{itemize}
なお、チューリングマシンの論理的表現の要素は以下となる。
\begin{itemize}
\item $M$: 状態を表す有限集合
\item $T$: 記号集合
\item $\delta$: $\delta(m,t) = (m', t', h)$: 遷移関数、ただし、$m,m' \ni M$、$t,t' \ni T$、$h$: テープの移動方向
\end{itemize}
つまり、マシンは現在の状態($m$、$t$)を読み、遷移関数の定義に従い次の状態($m'$、$t'$)を作り、どちらかに移動し、これを繰り返す。
ただし、以上は最も簡単な構造であり、より複雑な多次元テープと複数デバイスのモデルも構築することができる。
実際の計算機は万能チューリングマシン、つまり、他のチューリングマシンをエミュレート可能なチューリングマシンとして設計実装されており、基本的には上記の考えを拡張したものである。

\subsection{セルオートマトン}
\subsection{プログラミングとは}

\begin{breakbox}$\Rightarrow$
テスト用。
\end{breakbox}
