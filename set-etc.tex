データとは、計算機が扱うことのできる情報の形式と考えることができる。
この意味では、データを理解するためには計算機の構造をある程度理解する必要があることを述べておく。
このことは「データ処理とは」の章でもう少し詳しく解説するが、本書では端的には「デジタル電子計算機が扱うことのできるバイト列またはビット列」を指す。
\subsection{データ型}
一般的にデータ型とはデジタル電子計算機が扱うデータフォーマットのことと考えてよい。データ型には幾つかの抽象化レベルがあるが、最下層としては単なるビット列をどのように解釈するかである。すなわち、例えば以下のようなものである。
\begin{itemize}
\item 整数型
\item 小数(浮動小数点数)型
\item 文字型
\end{itemize}
\begin{breakbox}
$\Rightarrow$ 電子計算機以外にどのようなタイプの計算機があるか調べてみよう。
\end{breakbox}

\subsection{文字と文字列}
ここで、文字と文字列、数字について述べておこう。
文字は文字コードによって解釈されるが、文字コードはさらにエンコードされている。
UTF8とは、ユニコードをエンコードする一つの方法である。数字とは、数を表すための文字であり、数値とは異なる。
数値とは前述したように、整数や小数という概念そのもの、あるいは計算機上で数値として取り扱うビット列のことである。
文字列とは文字コードまたは文字エンコードの並らびである。
\begin{breakbox}
$\Rightarrow$ このことをCの\shortstack[l]{sscanf()}などで確認しよう。\\
$\Rightarrow$ リトルエンディアンとは何か調べてみよう。
\end{breakbox}

\subsection{配列、リスト、ツリー、グラフ}
一般的には、データを単体(datam)として扱うことは少なく、集合として扱うことが多い。
言わばデータセットである。
この場合の集合(=セット)とは文字通りの集合として扱う場合もあるが、データ列(すなわち並び順に意味がある)である。

\subsection{均質化グラフ}
