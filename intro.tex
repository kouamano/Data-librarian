本書はもともと、筑波大学情報学群の学部生を対象とした授業テキストであったものを、ライブラリアンのトレーニング向けの内容に改編したものである。
本書が対象とする読者は、現職の図書館員、または、相当の業務(研究支援等)を行う職員の中で、さらにデータライブラリアンとしての専門性を目指す人々である。
広くは研究所や大学の情報及び図書に関わる部門において研究データを取り扱う業務に従事する人々を対象としている。
前提としている知識・技術としては、すでに図書館の業務に十分に慣れており、IT技術水準は、コマンドラインシェルを扱え、簡単なプログラム(スクリプト)が組める程度である。
例えば、UNIXユーザーならbash、WindowsユーザーならPowershellを利用し、複数のプログラムを呼び出し、パイプライン化を行うようなことがすでに身についていることを前提としている。
また、データベースの知識があることが望ましい(利用経験はなくて良い)。
そのような読者にとっては、本書が、すでに身につけているデータ処理技術をさらに伸ばすトレーニングのきっかけとなるであろう。

本書は「一冊で終わる」本ではない。エッセンスのみを凝縮させたような、いわば、学習の起点となるような本を目指した。その目的から言えば深い理解を目指しているわけでもなく、研究者とデータについて語り合える最低限の計算機とデータの知識・技術を提示しているにすぎない。しかしながら、あるいはそれだけに、バランスの良い入門書としての存在を目指した。

トレーニングである以上、本書では各所で読者に実践を勧める記述がある。
\begin{breakbox}$\Rightarrow$
このようなboxがあるときは本書を離れ実践してみよう。
\end{breakbox}
\noindent{}スキルを身につけるためには、ぜひ、実践していただきたい。
当然、アプリケーションのインストールなどの基本的なPCのスキルは身についているものとして話を進めるので、必要なツールを判断し、適宜インストールして欲しい。
特に本書ではC言語を引き合いに出すことが多いので、C言語にはある程度慣れていただきたい。
