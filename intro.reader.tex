本書が対象とする読者は、現職の図書館員、または、相当の業務(研究支援等)を行う職員である。
広くは研究所や大学の情報及び図書に関わる部門において研究データを取り扱う業務に従事するものである。
前提としている知識・技術として、コマンドラインシェルを扱え、簡単なプログラム(スクリプト)が組める。
例えば、UNIXユーザーならbash、WindowsユーザーならPowershellを利用し、複数のプログラムを呼び出し、パイプライン化を行うようなことがすでに身についていることを前提としている。
また、データベースの知識があることが望ましい(利用経験はなくて良い)。
本書は「一冊で終わる」本ではない。エッセンスのみを凝縮させたような、いわば、学習の起点となるような本を目指した。その目的から言えば深い理解を目指しているわけでもなく、研究者とデータについて語り合える最低限の計算機とデータの知識・技術を提示しているにすぎない。しかしながら、あるいはそれだけに、バランスの良い、一種の入門書としての存在となることを目指した。
