\section{データサイエンティストとデータライブラリアン}
データライブラリアンと同様にデータに深く関わる職としてデータサイエンティストと言われるものがある。
近年、研究におけるデータ管理・公開の重要性が強く認識されるようになり、これらの職種に対して注目と期待が集まっているが、
一つの差別化の考えとしては、
データサイエンティストはあくまでも研究者であり、研究対象がたまたまデータであるに過ぎないと考えることができる。
この意味において、他の研究者と同様、常にライブラリアンの助けを必要としている。


\section{想定するデータライブラリアンの仕事}
データライブラリアンは、研究者が研究活動に専念できるように様々なサポートを行なうべき存在である。
雑誌契約やレファレンス、図書受け入れ作業などと同格な業務として、データ利用契約、データ検索、登録データのチェック、公開データの構造化、これらに関わるシステムの設計、ツールの作成などといった作業を行うことになるであろうことが予想され、業務の目的としては何ら今までのライブラリアンの仕事と変わらない。

ただ、技術面に注目すれば、このような作業を行うには、各分野におけるデータの性質に関する知識があることに加え、データチェックのための技能、データを構造化するための知識やデータ構造の解析技術などが身についている必要がある。
また、このような様々な形態のデータを処理するにあたり、毎回対応するスクリプティングやプログラミングを行うことは非効率的であり、これらのデータを効率的に扱うことができる高機能なツールは必須である。
しかしながら、逆説的ではあるが、高機能なツールでは処理が重すぎて利用が困難な場合もある。
つまり上記の業務を効率的に行うには、いずれの場合にも対応できるように、それぞれのツールに対する知識が必要となる。
本書によって、これらのツールや機能を使いこなすための、データに対する知識とデータ処理の基本的な考え方が身につくことを期待している。
